A \emph{category} $\mathcal{C}$ consists of the following data:
\begin{enumerate}
\item a class $\operatorname{ob}(\mathcal{C})$ of objects (of $\mathcal{C}$)
\item for each ordered pair $(A,B)$ of objects of $\mathcal{C}$, a collection (we will assume it is
 a set) $\hom(A,B)$ of morphisms from the domain $A$ to the codomain $B$
\item a function $\circ:\hom(A,B)\times\hom(B,C)\to\hom(A,C)$ called composition.
\end{enumerate}

We normally denote $\circ(f,g)$ by $g \circ f$ for morphisms $f,g$. The above data must satisfy the following axioms: for objects $A,B,C,D$,

\textbf{A1}: $\hom(A,B) \cap \hom(C,D)=\emptyset$ whenever $(A,B)\neq (C,D)$, i.e. the intersection is non-trivial only when $A=C$ and $B=D$.

\textbf{A2}: (Associativity) if $f \in \hom(A,B)$, $g\in\hom(B,C)$ and $h\in\hom(C,D)$, $h\circ (g\circ f)=(h\circ g)\circ f$

\textbf{A3}: (Existence of an identity morphism) for each object $A$ there exists an identity morphism $ {}id_{A}\in\hom(A,A)$ such that for every $f\in\hom(A,B)$, $f\circ id_{A}=f$ and $ {}id_{A}\circ g=g$ for every $g \in \hom(B,A)$.

Some examples of categories:
\begin{itemize}
\item \textbf{0} is the empty category with no objects or morphisms, \textbf{1} is the category with one object and one (identity) morphism.
\item If we assume we have a universe $U$ which contains all sets encountered in ``everyday'' mathematics,
\textbf{Set} is the category of all such small sets with morphisms being set functions
\item \textbf{Top} is the category of all small topological spaces with morphisms continuous functions
\item \textbf{Grp} is the category of all small groups whose morphisms are group homomorphisms
\end{itemize}

\textbf{Remark}.  If $\hom(A,B)$ in the second condition above is not required to be a set (but a class), we usually call $\mathcal{C}$ a \emph{large category}.

\subsubsection{Initial and terminal objects}
An {\em initial object} in a category $\mathcal{C}$ is an object $A$ in $\mathcal{C}$ such that, for every object $X$ in $\mathcal{C}$, there is exactly one morphism $A \longrightarrow X$. A {\em terminal object} in a category $\mathcal{C}$ is an object $B$ in $\mathcal{C}$ such that, for every object $X$ in $\mathcal{C}$, there is exactly one morphism $X \longrightarrow B$. A {\em zero object} in a category $\mathcal{C}$ is an object $0$ that is both an initial object and a terminal object. All initial objects (respectively, terminal objects, and zero objects), if they exist, are isomorphic in $\mathcal{C}$.

\subsubsection{Functors}
Given two categories $\mathcal{C}$ and $\mathcal{D}$, a covariant\ {\em functor} $T:\mathcal{C}\to\mathcal{D}$ consists of an assignment for each object $X$ of $\mathcal{C}$ an object $T(X)$ of $\mathcal{D}$ (i.e. a ``function'' $T:{\rm Ob}(\mathcal{C})\to{\rm Ob}(\mathcal{D})$) together with an assignment for every morphism $f\in{\rm Hom}_{\mathcal{C}}(A,B)$, to a morphism $T(f)\in{\rm Hom}_{\mathcal{D}}(T(A),T(B))$, such that:
\begin{itemize}
\item $T(1_A) = 1_{T(A)}$ where $1_X$ denotes the identity morphism on the object $X$ (in the respective category).
\item $T(g \circ f) = T(g)\circ T(f)$, whenever the composition $g\circ f$ is defined.
\end{itemize}

A contravariant functor $T :\mathcal{C}\to\mathcal{D}$ is just a covariant functor $T:\mathcal{C}^{\rm op}\to\mathcal{D}$ from the opposite category.  In other words, the assignment reverses the direction of maps.  If $f\in{\rm Hom}_{\mathcal{C}}(A,B)$, then $T(f)\in{\rm Hom}_{\mathcal{D}}(T(B),T(A))$ and $T(g\circ f) = T(f)\circ T(g)$ whenever the composition is defined (the domain of $g$ is the same as the codomain of $f$).

Given a category $\mathcal{C}$ and an object $X$ we always have the functor $T : \mathcal{C}\to{\bf Sets}$ to the category of sets defined on objects by $T(A) = {\rm Hom}(X, A)$.  If $f : A \to B$ is a morphism of $\mathcal{C}$, then we define $T(f) : {\rm Hom}(X,A)\to {\rm Hom}(X,B)$ by $g\mapsto f\circ g$.  This is a covariant functor, denoted by ${\rm Hom}(X,-)$.

Similarly, one can define a contravariant functor ${\rm Hom}(-,X) :\mathcal{C}\to{\bf Sets}$.

\subsubsection{Natural transformations}
Let $\mathcal{C}$ and $\mathcal{D}$ be categories, and let 
$S,T:\mathcal{C}\to\mathcal{D}$ be covariant functors. Then suppose
that for every object $A$ in $\mathcal{C}$ one has a morphism 
$\eta_A :  S(A) \to T(A) $ in $\mathcal{D}$ such that for every morphism 
$\alpha: A \to B$ in $\mathcal{C}$ the following
$$
\xymatrix@+=4pc{S(A) \ar[d]_{S(\alpha)} \ar[r]^{\eta_A} & T(A) \ar[d]^{T(\alpha)} \\
S(B) \ar[r]^{\eta_{B}} & T(B)
}
$$
is commutative.  Then we variously write 
$$
\eta: S \dot{\to} T \quad\mbox{ or }\quad \eta: S\Rightarrow T\quad \mbox{ or } \quad \eta:S\to T
$$
and call $\eta$ a \emph{natural trasformation} from $S$ to $T$.

One may think of a natural transformation $\eta:S\to T$ as a `function' from the class of objects of $\mathcal{C}$ to the class of morphisms of $\mathcal{D}$.

As a first example, for every functor $S:\mathcal{C}\to \mathcal{D}$, we can associate the natural transformation $1_S: S\to S$ (the \emph{identity natural transformation} on $S$) that assigns every object $A$ of $\mathcal{C}$, the corresponding identity morphism $1_{S(A)}$.

Natural transformations are composed in a similar manner to morphisms, but they are nevertheless defined as correspondences between both objects and morphisms as shown in the square commutative diagram depicted above. 

More precisely, given three functors $R,S,T:\mathcal{C}\to \mathcal{D}$, and two natural transformations, $\tau:R\to S$ and $\eta:S\to T$, we define the composition of $\tau$ with $\eta$, written $\eta \bullet \tau$, as a class of morphisms in $\mathcal{D}$ given by 
$$(\eta\bullet \tau)_A := \eta_A\circ \tau_A,$$ for every object $A$ in $\mathcal{C}$.
It is easy to see that $\eta\bullet \tau$ is a natural transformation, since we may ``compose'' two commutative squares and obtain a third one:
$$
\xymatrix@+=4pc{R(A) \ar[d]_{R(\alpha)} \ar[r]^{\tau_A}  & S(A) \ar[d]_{S(\alpha)} \ar[r]^{\eta_A} & T(A) \ar[d]^{T(\alpha)} \ar@{}[dr]|{=} &  
R(A) \ar[d]_{R(\alpha)} \ar[r]^{\eta_A \circ \tau_A} & T(A) \ar[d]^{T(\alpha)} 
\\
R(B) \ar[r]^{\tau_{B}} & S(B) \ar[r]^{\eta_{B}} & T(B) & 
R(B) \ar[r]^{\eta_B \circ \tau_B} & T(B)
}
$$
It is easy to see that the composition ``operation'' on natural transformations is associative:
$$(\zeta\bullet \eta)\bullet \tau = \zeta\bullet (\eta \bullet \tau)$$
for natural transformations $\tau:R\to S$, $\eta:S\to T$, and $\zeta:T\to U$.  In addition, any identity natural transformation acts as a compositional identity: if $\tau:R\to S$ and $\eta:S\to T$, then $$1_S\bullet \tau=\tau \qquad\mbox{ and }\qquad \eta \bullet 1_S = \eta.$$

\subsubsection{Adjoint functors}
Let $\mathcal{C}$ and $\mathcal{D}$ be (small) categories, and let $T:\mathcal{C} \to \mathcal{D}$ and $S:\mathcal{D} \to \mathcal{C}$ be covariant functors. $T$ is said to be a \emph{left adjoint functor} to $S$ (equivalently, $S$ is a \emph{right adjoint functor} to $T$) if there is a natural equivalence
\[
\nu\colon \Hom_{\mathcal{D}}(T(-),-) \overset{\cdot}{\longrightarrow} \Hom_{\mathcal{C}}(-,S(-)).
\]
Here the functor $\Hom_{\mathcal{D}}(T(-),-)$ is a bifunctor $\mathcal{C}\times\mathcal{D}\to\mathbf{Set}$ which is contravariant in the first variable, is covariant in the second variable, and sends an object $(C,D)$ to $\Hom_{\mathcal{D}}(T(C),D)$.  The functor $\Hom_{\mathcal{C}}(-,S(-))$ is defined analogously.

This definition needs additional explanation.  Essentially, it says that for every object $C$ in $\cal{C}$ and every object $D$ in $\cal{D}$ there is a function 
\[
\nu_{C,D} \colon \Hom_{\mathcal{D}}(T(C),D) \overset{\sim}{\longrightarrow} \Hom_{\mathcal{C}}(C,S(D)) 
\]
which is a natural bijection of hom-sets.  Naturality means that if $f\colon C'\to C$ is a morphism in $\mathcal{C}$ and $g\colon D\to D'$ is a morphism in $\mathcal{D}$, then the diagram
\[\xymatrix{
\Hom_{\mathcal{D}}(T(C),D)\ar[dd]_{(Tf,g)}\ar[rr]^{\nu_{C,D}} &&
\Hom_{\mathcal{C}}(C,S(D))\ar[dd]^{(f,Sg)} \\ && \\
\Hom_{\mathcal{D}}(T(C'),D')\ar[rr]^{\nu_{C',D'}} &&
\Hom_{\mathcal{C}}(C',S(D')) \\
}\] 
is a commutative diagram.  If we pick any $h:T(C)\to D$, then we have the equation $$Sg\circ \nu_{C,D}(h)\circ f= \nu_{C',D'}(g\circ h\circ Tf).$$

If $T:\mathcal{C}\to\mathcal{D}$ is a left adjoint of $S:\mathcal{D}\to \mathcal{C}$, then we say that the ordered pair $(T,S)$ is an \emph{adjoint pair}, and the ordered triple $(T,S,\nu)$ an \emph{adjunction} from $\mathcal{C}$ to $\mathcal{D}$, written $$(T,S,\nu):\mathcal{C}\to \mathcal{D},$$ where $\nu$ is the natural equivalence defined above.  

An adjoint to a functor is in some ways like an inverse (as in the case of an adjoint matrix); often formal properties about a functor lead to formal properties of its adjoint (for example the right adjoint to a left-exact functor takes injectives to injectives).  An adjoint to any functor is unique up to natural isomorphism.

\subsubsection{Exponential objects}
Let $A,B$ be objects in a category with finite products $\mathcal{C}$.  An object $E$ in $\mathcal{C}$ is called an \emph{exponential object} from $A$ to $B$ if it satisfies the following conditions:
\begin{itemize}
\item there is a morphism $f:E\times A\to B$, called an \emph{evaluation morphism}
\item for any morphism $g:C\times A\to B$, there is a unique morphism $h:C\to E$ such that $f\circ (h\times 1_A)=g$, where $h\times 1_A:C\times A\to E\times A$ is the product morphism of $h$ and the identity morphism on $A$.
\end{itemize}
The two conditions can be summarized by the following commutative diagram:
\begin{center}
$
\xymatrix@R-=20pt{
E\times A\ar[dr]^f\\
&B\\
C\times A\ar[ur]_g\ar[uu]^{h\times 1_A}
}
$
\end{center}
where $h$ is uniquely determined by $g$.  It is easy to see that any two exponential objects from $A$ to $B$ are isomorphic, hence the existence of an exponential object between two objects is a universal property.  We may write $B^A (\cong E$ above) \emph{the} exponential object from $A$ to $B$.

For example, in the category of sets, $\textbf{Set}$, where products exist between pairs of objects (sets), the exponential from $A$ to $B$ is the set $B^A$, which is defined as the set of all functions from $A$ to $B$.  The evaluation morphism is the function $ev: B^A\times A\to B$ given by $ev(f,a)=f(a)$, where $f\in B^A$ and $a\in A$.  If $g:C\times A\to B$ is any function, then we define $h:C\to B^A$ by $h(c)(a)=g(c,a)$.  Then $ev\circ (h\times 1_A)(c,a)=ev(h(c),a)=h(c)(a)=g(c,a)$, and $ev$ is universal (in the sense of the second condition above).

Since each $h$ is uniquely determined by $g$ in the above definition, and conversely every $h$ determines a $g$ by the formula $g=f\circ (h\times 1_A)$, we have a bijection 
$$
\hom(C\times A,B)\cong \hom(C,B^A).
$$
If an exponential object exists between every pair of objects in category $C$ with finite products, then we say that $C$ \emph{has exponentials}.  According to the bijection above, we see that the functor $\cdot\times A:\mathcal{C}\to \mathcal{C}$ has a right adjoint, namely $\cdot ^A:\mathcal{C}\to\mathcal{C}$, called the \emph{exponential functor}.
