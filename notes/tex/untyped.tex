%!TEX root = ../notes.tex
\section{The type-free lambda calculus and its categorical semantics}
The type-free lambda calculus can be seen as a special case of the simply-typed lambda calculus in which the set of basic types is restricted to contain only a single type. With the typing constraint imposed by having multiple types removed, it should be possible to perform self-application as in the term $\lambda x. xx$. If the second occurence of $x$ in this term has type $D$ and the whole term $xx$ has type $D$ then the first occurrence of $x$ in the term $xx$ must be construable as having type $[D \rightarrow D]$. A presentation of the type-free theory, $\mathbb{T}$, can then be given in a manner analogous to that of the simply typed lambda calculus~\cite{Awodey2000}:
\begin{enumerate}
\item{Types:}
\begin{align*}
&\mbox{one basic type: } D\\
\end{align*}
\item{Terms:}
\begin{align*}
&\mbox{variables: } x,y,z, \ldots \colon D\\
&\mbox{constants: } i \colon [D \rightarrow D] \rightarrow D,\,\, r \colon D \rightarrow [D \rightarrow D]\\
\end{align*}
\item{Equations:}
\begin{align*}
            \lambda x. ri(x) &= \lambda x.x\\
            \lambda x. ir(x) &= \lambda x.x
\end{align*}
\end{enumerate}
On this basis, we are led to the requirement that
$$
[D \rightarrow D] \cong D,
$$
which is a particular instantiation of a more general phenomenon referred to as a recursive domain equation. The process by which $[D \rightarrow D]$ is formed in the first place is via the internal Hom bifunctor $F \equiv [-,-] \colon \mathcal{C}^{op} \times \mathcal{C} \rightarrow \mathcal{C}$ on a category $\mathcal{C}$, which thereby restricts consideration for the purpose of finding denotational semantics for the type-free $\lambda$-calculus to categories $\mathcal{C}$ that are closed since this is a necessary condition for it to posses the necessary structure to define the internal Hom bifunctor. In this light, the solution $D$ to the equation $[D,D] \cong D$ is a fixed point $F(D,D) \cong D$ of the internal Hom bifunctor on some category $\mathcal{C}$.

Given these restrictions, it is obvious that it will not be sufficient in the search for non-trivial models of the type-free theory of the $\lambda$-calculus to restrict consideration to the category $\mathbf{Sets}$ because in such a category $F$ has a fixed point $D$ only when $D$ is the one element set: $D = \{*\}$. However, $\mathbb{T}$ does not necessarily prove $\lambda y.ir(y)=\lambda y.y$, which would be true for $D = \{*\}$. Because something provable in $\mathbf{Sets}$ is not necessarily provable in the type-free $\lambda$-calculus there is no sense in which semantics in the category $\mathbf{Sets}$ could be complete with respect to the type-free $\lambda$-calculus.

We will very briefly sketch the solution to this particular problem and refer to other sources for caveats and details~\cite{Barendregt1985,Smyth1982,Freyd1990,Abramsky1995,Cattani2007}. In order to remedy this situation Scott eventually introduced \emph{domains}, which can be viewed as \emph{continuous lattices} (this construction appears with minor modifications with respect to the purpose of the present exposition in terms of various other closely related kinds of objects and their associated categories such as complete partial orders and complete lattices among others). A continuous lattice is a complete \todo{Include appropriate definitions from lattice theory (e.g. lattice, meet, join, completeness, etc.)} lattice $D$ such that
$$
\forall d \in D, d = \bigvee \left\{ \bigwedge U | d \in U, U \mbox{ Scott-open}, U \subseteq D \right\}
$$
where \emph{Scott-open} refers to a subset $U$ of a domain $D$ that is upward closed in the sense that if $\bigvee \Delta \in U$ for a directed subset $\Delta \subseteq D$ then $U \cap \Delta \neq \emptyset$ and Scott-continuous functions are required to be monotonic and preserve least upper bounds of directed subsets of their domains. These continuous lattices can be equivalently characterized in topological terms as $T_0$-spaces in which every continuous function $f \colon P \rightarrow D$ from a subspace $P \subseteq S$ can be extended to a Scott-continuous function $\hat{f} \colon S \rightarrow D$.

Domains of this form taken as objects and Scott-continuous functions between them taken as morphisms form cartesian closed categories where the internal-hom (i.e. exponential) objects were naturally themselves not sets but complete lattices of Scott-continuous functions. Since these form a cartesian closed category they could be used as described above to provide semantics for the simply-typed $\lambda$-calculus. However, categories of domains can also be constructed that have reflexive objects, $D_\infty$, which solve the recursive domain equation $F(D_{\infty},D_{\infty})=D_{\infty}$ that arises naturally as described above in the analysis of the type-free $\lambda$-calculus since, there, objects serve both as arguments and as functions that can be applied to such arguments.

The way in which such solutions $D_\infty$ are constructed is via a generalization of the least fixed-point of a continuous function. A continous function $f \colon D \rightarrow D$ may have fixed-points $x$ such that $f(x)=x$. The least fixed-point for such a continuous function $f$ is then given by
$$
fix(f) = \bigvee_{n \in \omega} f^n(\bot)
$$
Now, we would like to extend this notion to the internal-hom bifunctor $F \equiv [-,-] \colon \mathcal{C}^{op} \times \mathcal{C} \rightarrow \mathcal{C}$ for $\mathcal{C}$ a category of continuous lattices and Scott-continuous functions. Given domains $D$ and $E$ as objects in such a category, we can define an \emph{embedding-projection pair} from $D$ to $E$ as a pair of Scott-continuous functions $i \colon D \rightarrow E$ and $r \colon E \rightarrow D$ such that $r \circ i = id_D$ and $i \circ r \leq id_E$ where the relation of order on functions is given pointwise. Reflexive domains are given by inverse limits (or projective limits, which are specializations of limits, as opposed to colimits, in category theory) of sequences of projections $r_n$ given by:
\begin{align*}
&D_0 = \mbox{ some object, $D$, in } \mathcal{C},\\
&D_1 = [D_0 \rightarrow D_0],\\
&D_{n+1} = [D_n \rightarrow D_n],\\
&(r_n \colon D_{n+1} \rightarrow D_n)_{n \in \omega}.
\end{align*}
The fact that a suitable choice for the initial embedding-projection pair $\langle i_0, r_0 \rangle$ ultimately requires that the limit of the above sequence of projections coincide with the colimit of the sequence of embeddings
$$
(i_n \colon D_{n} \rightarrow D_{n+1})_{n \in \omega}.
$$
It is eventually possible to obtain
$$
\lim_{\leftarrow} (D_n, r_n) \cong D_\infty \cong \lim_{\rightarrow} (D_n, i_n).
$$
and this particular construction of $D_{\infty}$ provides a fixed-point $\text{FIX}(F)$ so that $F(D_{\infty},D_{\infty}) = D_{\infty}$ solves the given recursive domain equation associated to the untyped $\lambda$-calculus. This allows us to interpret the terms of the untyped lambda calculus as being the objects and morphisms $D_{\infty}$ and $[D_{\infty},D_{\infty}]$ respectively, which are now seen to be one and the same since $D_{\infty} \cong D_{\infty}^{D_{\infty}}$.

