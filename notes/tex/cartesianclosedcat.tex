A category $\mathcal{C}$ with finite products is said to be \emph{Cartesian closed} if each of the following functors has a right adjoint
\begin{enumerate}
\item $\textbf{0}:\mathcal{C}\to \textbf{1}$, where $\textbf{1}$ is the trivial category with one object $0$, and $\textbf{0}(A)=0$
\item the diagonal functor $\delta: \mathcal{C}\to \mathcal{C}\times\mathcal{C}$, where $\delta(A)=(A,A)$, and
\item for any object $B$, the functor $(-\times B):\mathcal{C}\to\mathcal{C}$, where $(-\times B)(A)=A\times B$, the product of $A$ and $B$.
\end{enumerate}
Furthermore, we require that the corresponding right adjoints for these functors to be
\begin{enumerate}
\item any functor $\textbf{1}\to\mathcal{C}$, where $0$ is mapped to an object $T$ in $\mathcal{C}$.  $T$ is necessarily a terminal object of $\mathcal{C}$.  
\item the product (bifunctor) $(-\times -): \mathcal{C} \times \mathcal{C}\to \mathcal{C}$ given by $(-\times -)(A,B)\mapsto A\times B$, the product of $A$ and $B$.
\item for any object $B$, the exponential functor $(-^B):\mathcal{C}\to\mathcal{C}$ given by $(-^B)(A)=A^B$, the exponential object from $B$ to $A$.
\end{enumerate}

In other words, a Cartesian closed category $\mathcal{C}$ is a category with finite products, has a terminal objects, and has exponentials.  It can be shown that a Cartesian closed category is the same as a finitely complete category having exponentials.

Examples of Cartesian closed categories are the category of sets \textbf{Set} ( terminal object: any singleton; product: any Cartesian product of a finite number of sets; exponential object: the set of functions from one set to another)  the category of small categories \textbf{Cat} (terminal object: any trivial category; product object: any finite product of categores; exponential object: any functor category), and every elementary topos.